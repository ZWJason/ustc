% !TeX root = ./main.tex

\ustcsetup{
  title              = {面向嵌入式设备的神经网络压缩与加速方法},
  title*             = {Research on Optimization and Acceleration Methods of Convolutional Neural Networks on Embedded Platforms},
  author             = {翟文杰},
  author*            = {WenJie Zhai},
  speciality         = {软件工程},
  speciality*        = {Software Engineering},
  supervisor         = {朱宗卫~教授},
  supervisor*        = {Prof. Chao Wang},
  advisor            = {朱宗卫~高级工程师},
  advisor*           = {Senior Engineer Zongwei Zhu},
  % date               = {2017-05-01},  % 默认为今日
  professional-type  = {专业学位类型},
  professional-type* = {Professional degree type},
  % secret-level       = {秘密},     % 绝密|机密|秘密,注释本行则不保密
  % secret-level*      = {Secret},  % Top secret|Highly secret|Secret
  % secret-year        = {10},      % 保密年限
}

% 加载宏包
\usepackage{graphicx}
\usepackage{subfigure}
\usepackage{floatrow}
\usepackage{booktabs}
\usepackage{longtable}
\usepackage[ruled,linesnumbered]{algorithm2e}
\usepackage{siunitx}
\usepackage{xcolor}
\usepackage{amsthm}
\usepackage{color}
\usepackage{pifont}
\usepackage{multirow}

\usepackage{listings}
\lstset{basicstyle=\ttfamily}


% 配置图片的默认目录
\graphicspath{{figures/}}

% 用于写文档的命令
\DeclareRobustCommand\cs[1]{\texttt{\char`\\#1}}
\DeclareRobustCommand\pkg{\textsf}
\DeclareRobustCommand\file{\nolinkurl}


% hyperref 宏包在最后调用
\usepackage{hyperref}


\floatsetup[table]{capposition=top}
\newfloatcommand{capbtabbox}{table}[][\FBwidth]
