% !TeX root = ../main.tex

\begin{acknowledgements}

%时光如水,岁月如梭,转眼间,多姿多彩而又忙碌充实的硕士生涯即将落下帷幕。回首这个历程,在这三年的光阴里,科大不仅给了我学习知识的平台,也给我各方面的锻炼提供了机会。在这过程中,我收获了很多,不仅是知识上的丰富,也有能力和阅历的锻炼增长,马上将要走出校门,走上社会继续贡献自己。在这短短的时光,很多人给予了我无私帮助和辛勤的指导,无论是生活上还是学习上,即将毕业,我要把最真挚的感谢和祝福送给他们。
%
%首先要感谢帮助过我的老师们,您们兢兢业业,无私奉献,只为教育我们成长,丰富我们学识。在您们的带领下,才有我的今天。还有我的导师王超老师和朱宗卫老师,在三年的硕士生涯中,老师给予我无数的教导、指引、关爱和帮助。特别是朱宗卫老师,全程指导本论文的研究和撰写过程。从论文的开题到中期检查到最终上交,无论是在实验研究方法还是在论文写作上,每一步都有朱老师的细心指导,给了我许多宝贵的意见,令我获益良多。同时在这个过程中,老师锲而不舍的精神,认真负责的态度和创新性的思维都给了我极大的鼓励与启发,激励着我追求卓越,奋力拼搏。所以,非常荣幸能在朱老师的指导下,顺利完成了硕士论文撰写,为自己的硕士生涯画上一个圆满的句号。在此向您们表达我衷心的感谢!
%
%其次,还要感谢实验室各位同学和师弟们对我生活上和学习上的关心和帮助,感谢他们在我学习中对我的大力支持以及在论文写作中给予我的中肯建议,没有他们的无私帮助,我的论文实验不可能如此顺利的完成。再次真诚的感谢所有帮助过我的老师和同学和师弟们,在以后的生活中,我也一定会继续努力,不断前进。 
%
%最后,由衷的感谢我的家人和焦蕗屿同学对我的理解和支持,他们的鼓励、关怀和期待,是我为之奋斗,不断前行的动力。 
%
%通过撰写本次硕士论文,我不仅对专业知识有了更深一步的理解,也提高了自己独立思考问题,独立解决问题的能力,培养了我对于科学实验的严谨态度。在实验中,我明白做任何事情都不会一帆风顺的,所以要细心,更要有耐心。当然,由于自己的知识所限,在论文中肯定有考虑不到的地方,希望各位评审老师能够耐心指出,让我能够更进一步。所以,衷心的感谢评审本论文和参加论文答辩的专家、教授,谢谢你们无私奉献,在此向你们致以最诚挚的敬意!

\end{acknowledgements}
